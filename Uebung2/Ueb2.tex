\documentclass[11pt,titlepage]{article}

%Laenderspezifische Einstellungen bzgl. Rechtschreibung, Sonderzeichen und Kodierung
\usepackage[utf8]{inputenc}
\usepackage[naustrian]{babel}
\usepackage[T1]{fontenc}
\usepackage{titlesec}
\usepackage{graphicx}
%\usepackage{subcaption}

\usepackage{listings}
\usepackage{color}
\usepackage{courier}
\usepackage{matlab-prettifier}
\definecolor{light-gray}{gray}{0.85}
\lstset{
language=C++,
numbers=left,
style=Matlab-editor,
basicstyle=\mlttfamily,
breaklines=true,
backgroundcolor=\color{light-gray},
tabsize=2,
basicstyle=\footnotesize\ttfamily,
frame=single,
inputencoding=utf8,
extendedchars=true,
showstringspaces=false,
literate =
	{ä}{{\"a}}1
	{ö}{{\"o}}1
	{ü}{{\"u}}1
	{Ä}{{\"A}}1
	{Ö}{{\"O}}1
	{Ü}{{\"U}}1
	{ß}{{\ss}}1
	{ₙ}{{$_n$}}1
}

\def\ContinueLineNumber{\lstset{firstnumber=last}}
\def\StartLineAt#1{\lstset{firstnumber=#1}}

\usepackage[
	a4paper,
	top = 2cm,
	bottom = 2 cm,
	left = 2cm,
	right = 2cm,
	headheight = 15pt,
	includeheadfoot
	]{geometry}
\usepackage{fancyhdr}
\usepackage{amssymb}
\usepackage{amsmath}
\usepackage[english]{varioref}
\usepackage{hyperref}

\fancypagestyle{fancy}{
	\fancyhead[R]{Page \thepage}
	\fancyhead[L]{\leftmark}
	\renewcommand{\headrulewidth}{1.25pt}

	\fancyfoot[L]{\tiny{Alg. Meth. i.d. Num. - Uebung2 , created: \today}}
	\fancyfoot[R]{\tiny{Felix Dreßler (k12105003), Elisabeth Köberle (k12110408)}}
	\cfoot{}
	\renewcommand{\footrulewidth}{1.25pt}
}

\setlength{\headsep}{10mm}
\setlength{\footskip}{10mm}

\setlength{\parindent}{0mm}
\setlength{\parskip}{1.1ex plus0.25ex minus0.25ex}
\setlength{\tabcolsep}{0.2cm} % for the horizontal padding

\pagestyle{fancy}

\title{Algorithmische Methoden in der Numerik - Uebung2}
\author{Felix Dreßler (k12105003) \\ Elisabeth Köberle (k12110408)}
\date{\today} %Erstellungsdatum

\begin{document}
\maketitle

	\section{Aufgabe a - QRFact}
	
		\lstinputlisting[]{QRFact.m}
			
\newpage			
	\section{Aufgabe b - QRSolve}
	
	Unter Verwendung von den in \emph{Aufgabe c} berechneten \emph{Q} und \emph{R} wurde in \emph{Aufgabe b} der Vektor \emph{x} berechnet.
	
		\lstinputlisting[]{QRSolve.m}
		
\newpage
	\section{Aufgabe c}
		\subsection{CompR}
		
			\lstinputlisting[]{CompR.m}
			
		\subsection{CompQ}
		
			\lstinputlisting[]{CompQ.m}
			
\newpage
	\section{Tests}
		Aufgrund der besseren leserlichkeit wurde auf genauere Darstellung der Zahlen großteils verzichtet. Die Tests wurden dafür alle als Matlab Workspace gespeichert und beigelegt.
	
		\subsection{QRFact Tests}
		Im folgenden wird mit Matrizen der Größe 2x2, 4x2, 10x5 und 1000x100 getestet. 
			\subsubsection{2x2 Matrix}
				\begin{lstlisting}
					>> A1=randMatrix(2,2,2)
					
					A1 =
					
					0.5377   -2.2588
					1.8339    0.8622
					>> [B1,D1,p1,k1] = QRFact(A1)
					
					B1 =
					
					2.4487   -0.1918
					1.8339    4.8203
					
					
					D1 =
					
					-1.9111
					-2.4102
					
					
					p1 =
					
					1     2
					
					
					k1 =
					
					2
				\end{lstlisting}
			\subsubsection{4x2 Matrix}
				\begin{lstlisting}
					>> A2 = randMatrix(4,2,2)
					
					A2 =
					
					0.3188    3.5784
					-1.3077    2.7694
					-0.4336   -1.3499
					0.3426    3.0349
					
					>> [B2,D2,p2,k2] = QRFact(A2)
					
					B2 =
					
					1.7738    0.5881
					-1.3077   10.5562
					-0.4336   -0.6189
					0.3426    2.4573
					
					
					D2 =
					
					-1.4550
					-5.5823
					
					
					p2 =
					
					1     2
					
					
					k2 =
					
					2
				\end{lstlisting}
			\subsubsection{10x5 Matrix}
				\begin{lstlisting}
					A3 = randMatrix(10,5,5)
					
					A3 =
					
					0.7254    0.7172   -1.0689    0.3192   -1.2141
					-0.0631    1.6302   -0.8095    0.3129   -1.1135
					0.7147    0.4889   -2.9443   -0.8649   -0.0068
					-0.2050    1.0347    1.4384   -0.0301    1.5326
					-0.1241    0.7269    0.3252   -0.1649   -0.7697
					1.4897   -0.3034   -0.7549    0.6277    0.3714
					1.4090    0.2939    1.3703    1.0933   -0.2256
					1.4172   -0.7873   -1.7115    1.1093    1.1174
					0.6715    0.8884   -0.1022   -0.8637   -1.0891
					-1.2075   -1.1471   -0.2414    0.0774    0.0326
					
					>> [B3,D3,p3,k3] = QRFact(A3)
					
					B3 =
					
					3.7619   -0.4257    1.5019   -0.9862   -0.0047
					-0.0631   -1.2958    0.6194    0.5266   -4.0112
					0.7147    0.4595   -6.4680    0.5611    0.2229
					-0.2050    4.5881    0.7601    0.5911    1.4667
					-0.1241    0.1682    0.5374   -1.8159   -0.8096
					1.4897   -0.1315   -0.0490    0.0594    0.8503
					1.4090    0.0244    2.2497    0.0118    0.2274
					1.4172   -0.0580   -1.3203    0.8758    1.5730
					0.6715    0.0407    0.6771   -1.2240   -0.8732
					-1.2075   -1.0379   -0.9361    0.5721   -0.3556
					
					
					D3 =
					
					-3.0365
					2.8775
					3.9304
					-2.4170
					1.6228
					
					
					p3 =
					
					1     5     3     2     4
					
					
					k3 =
					
					5
				\end{lstlisting}
			\subsubsection{1000x100 Matrix}
				Dieser Test wird aus übersichtlichkeitsgründen nicht im PDF angeführt. Beiliegend ist jedoch die Matlab Workspace-Datei \emph{TestsQRFact.mat} in der alle Tests mit Inputs und Outpus abgespeichert sind.
				
\newpage
		\subsection{QRSolve Tests}
			\subsubsection{Test 1}
				\begin{lstlisting}
					>> b1 = randMatrix(3,1,1)
					
					b1 =
					
					1.4188
					-1.9819
					-0.2029
					
					>> A1 = randMatrix(3,3,3)
					
					A1 =
					
					-1.2212   -1.7193   -1.2536
					-0.0602    0.1326   -1.8723
					-1.6034   -0.2888   -0.8403
					
					>> [B1,D1,p1,k1] = QRFact(A1)
					
					B1 =
					
					-3.2377    1.2670    1.4834
					-0.0602    0.1428   -3.7142
					-1.6034    2.3929    0.5151
					
					
					D1 =
					
					2.0164
					1.8928
					-1.1964
					
					
					p1 =
					
					1     3     2
					
					
					k1 =
					
					3
					
					>> x1 = QRSolve(B1, D1, p1, k1, b1)
					
					x1 =
					
					-0.1169
					-1.4422
					0.9601
					
					>> A1 * x1
					
					ans =
					
					1.4188
					-1.9819
					-0.2029
			
					x1alt = linsolve(A1,b1)
					
					x1alt =
					
					-0.1169
					-1.4422
					0.9601
					
					>> format long
					>> F_abs = norm(x1 - x1alt)
					
					F_abs =
					
					4.284169974453670e-16
					
					>> F_rel = F_abs / norm(x1)
					
					F_rel =
					
					2.467087919033295e-16
				
				\end{lstlisting}
			\subsubsection{Test 2}
			Test wurde mit einer 100x100 Matrix durchgeführt, die jedoch nur Rang 99 hat. Wie erwartet ist das Ergebnis der 0 Vektor.
				\begin{lstlisting}
					A2 = randMatrix(100,100,99)
					
					...
					
					[B2,D2,p2,k2] = QRFact(A2)
					
					...
					
					b2 = randMatrix(100,1,1)
					
					...
					
					x2 = QRSolve(B2,D2,p2,k2,b2)
					
					x2 =
					
					[]
				\end{lstlisting}
			\subsubsection{Test 3}
				\begin{lstlisting}
					>> A3 = randMatrix(5,5,5)
					
					A3 =
					
					1.6703    0.9527   -0.5493    1.1881   -0.3618
					-1.5417    0.1314   -0.3175   -1.2128   -0.4264
					-0.2720   -1.7419   -0.5827   -0.3812    0.3871
					0.3416    0.6678   -0.1642    1.3227    0.9028
					0.4844    0.8982   -0.9351    0.3853    0.7178
					
					>> b3 = randMatrix(5,1,1)
					
					b3 =
					
					1.7361
					-1.0088
					-0.1800
					-2.0265
					0.4089
					
					>> [B3,D3,p3,k3] = QRFact(A3)
					
					B3 =
					
					4.0350   -1.0681    0.3292   -1.9437   -0.2553
					-1.5417   -0.0763   -0.4769    0.6954   -1.7814
					-0.2720   -0.1612   -1.8231    0.3048    0.3800
					0.3416    3.1299   -0.1800   -0.4438    0.9118
					0.4844    1.6687   -0.9020   -1.2232    0.7306
					
					
					D3 =
					
					-2.3647
					1.3144
					1.1435
					-2.0098
					0.6116
					
					
					p3 =
					
					1     5     3     2     4
					
					
					k3 =
					
					5
					
					>> x3 = QRSolve(B3,D3,p3,k3,b3)
					
					x3 =
					
					2.3944
					0.1683
					-0.0477
					-2.1173
					-0.1821
					
					>> x3alt = linsolve(A3,b3)
					
					x3alt =
					
					2.3944
					0.1683
					-0.0477
					-2.1173
					-0.1821
					
					>> format long
					>> F_abs = norm(x3 - x3alt)
					
					F_abs =
					
					7.157831469485814e-16
					
					>> F_rel = F_abs / norm(x3)
					
					F_rel =
					
					2.232497222110345e-16
				\end{lstlisting}
			\subsubsection{Test 4}
				\begin{lstlisting}
					>> A4 = randMatrix(100,100,100)
					
					...
					
					>> b4 = randMatrix(100,1,1)
					
					...
					
					>> [B4,D4,p4,k4] = QRFact(A4)
					
					...
					
					>> x4 = QRSolve(B4,D4,p4,k4,b4)
					
					...
					
					>> x4alt = linsolve(A4,b4)
					
					...
					
					>> F_abs = norm(x4 - x4alt)
					
					F_abs =
					
					1.951423563331391e-12
					
					>> F_rel = F_abs / norm(x4)
					
					F_rel =
					
					4.991915427983470e-14
						
				\end{lstlisting}

\newpage
		\subsection{CompR, CompQ Tests}
		Für diese Tests wurden die Matrizen aus den Tests für QRFact verwendet.
			\subsubsection{2x2 Matrix}
				\begin{lstlisting}
					>> R1 = CompR(B1,D1,p1,k1)
					
					R1 =
					
					-1.9111   -0.1918
					0   -2.4102
					
					>> Q1 = CompQ(B1,p1,k1)
					
					Q1 =
					
					-0.2813    0.9596
					-0.9596   -0.2813
					
					>> F_rel1 = norm(Q1*R1-A1(:,p1))/norm(A1)
					
					F_rel1 =
					
					9.1373e-17
					
					>> [Q1alt,R1alt,e1alt] = qr(A1,'vector')
					
					Q1alt =
					
					-0.9343    0.3566
					0.3566    0.9343
					
					
					R1alt =
					
					2.4178    0.1516
					0    1.9051
					
					
					e1alt =
					
					2     1
					
					>> F_rel1alt = norm(Q1alt*R1alt-A1(:,e1alt))/norm(A1)
					
					F_rel1alt =
					
					4.5686e-17
				\end{lstlisting}
			\subsubsection{4x2 Matrix}
				\begin{lstlisting}
					>> R2 = CompR(B2,D2,p2,k2)
					
					R2 =
					
					-1.4550    0.5881
					0   -5.5823
					0         0
					0         0
					
					>> Q2 = CompQ(B2,p2,k2)
					
					Q2 =
					
					-0.2191   -0.6641    0.3896   -0.5993
					0.8987   -0.4014   -0.1764    0.0016
					0.2980    0.2732    0.8983    0.1723
					-0.2355   -0.5685    0.1011    0.7818
					
					>> F_rel2 = norm(Q2*R2-A2(:,p2))/norm(A2)
					
					F_rel2 =
					
					2.2812e-16
					
					>> [Q2alt,R2alt,e2alt] = qr(A2,'vector')
					
					Q2alt =
					
					-0.6375    0.2875    0.3544   -0.6208
					-0.4934   -0.8517   -0.1760    0.0118
					0.2405   -0.3250    0.9067    0.1202
					-0.5407    0.2937    0.1460    0.7746
					
					
					R2alt =
					
					-5.6132    0.1525
					0    1.4470
					0         0
					0         0
					
					
					e2alt =
					
					2     1
					
					>> F_rel2alt = norm(Q2alt*R2alt-A2(:,e2alt))/norm(A2)
					
					F_rel2alt =
					
					1.1206e-16
				\end{lstlisting}
			\subsubsection{10x5 Matrix}
				\begin{lstlisting}
					>> R3 = CompR(B3,D3,p3,k3)
					
					R3 =
					
					-3.0365   -0.0047    1.5019   -0.4257   -0.9862
					0    2.8775    0.6194   -1.2958    0.5266
					0         0    3.9304    0.4595    0.5611
					0         0         0   -2.4170    0.5911
					0         0         0         0    1.6228
					0         0         0         0         0
					0         0         0         0         0
					0         0         0         0         0
					0         0         0         0         0
					0         0         0         0         0
					
					>> Q3 = CompQ(B3,p3,k3)
					
					Q3 =
					
					-0.2389   -0.4223   -0.1141   -0.0500    0.2462   -0.3985   -0.4760   -0.4206   -0.1194    0.3345
					0.0208   -0.3869   -0.1529   -0.4998    0.5659    0.3017    0.1566    0.0997    0.3001   -0.2022
					-0.2354   -0.0028   -0.6587   -0.2846   -0.3437   -0.0747    0.4742   -0.1845   -0.1931    0.0990
					0.0675    0.5327    0.2562   -0.6769    0.0076   -0.0422   -0.1075   -0.0992    0.0718    0.3993
					0.0409   -0.2674    0.1093   -0.1438    0.0246    0.1095    0.0050    0.5769   -0.6716    0.3126
					-0.4906    0.1283   -0.0248    0.1385    0.0052    0.7773   -0.2156   -0.2018   -0.0914    0.1528
					-0.4640   -0.0792    0.5384    0.1049    0.1930   -0.1720    0.6139   -0.1323   -0.0023    0.1275
					-0.4667    0.3876   -0.3182    0.1397    0.3333   -0.2805   -0.1240    0.5259    0.1685    0.0337
					-0.2211   -0.3788    0.1182   -0.1030   -0.5470    0.0164   -0.1078    0.3184    0.5527    0.2551
					0.3977    0.0120   -0.2153    0.3572    0.2298    0.1346    0.2492   -0.0186    0.2448    0.6888
					
					>> F_rel3 = norm(Q3*R3-A3(:,p3))/norm(A3)
					
					F_rel3 =
					
					4.7262e-16
					
					>> [Q3alt,R3alt,e3alt] = qr(A3,'vector')
					
					Q3alt =
					
					-0.2513   -0.3896    0.1830   -0.0500    0.2462   -0.4375   -0.2117   -0.5951   -0.1097    0.2934
					-0.1903   -0.3632   -0.0734   -0.4998    0.5659    0.3197    0.0105    0.1914    0.2934   -0.1773
					-0.6923    0.0991   -0.0154   -0.2846   -0.3437    0.0292    0.5331   -0.0704   -0.1457    0.0190
					0.3382    0.4887    0.0276   -0.6769    0.0076   -0.0606   -0.0196   -0.1645    0.0729    0.3890
					0.0765   -0.2816    0.0012   -0.1438    0.0246    0.0270   -0.0534    0.4917   -0.7104    0.3790
					-0.1775    0.1565    0.4492    0.1385    0.0052    0.7380   -0.2776   -0.2266   -0.1132    0.1929
					0.3222   -0.1265    0.6258    0.1049    0.1930   -0.0459    0.6585    0.0184    0.0526    0.0371
					-0.4024    0.4515    0.3217    0.1397    0.3333   -0.3563   -0.1957    0.4559    0.1312    0.0992
					-0.0240   -0.3790    0.2493   -0.1030   -0.5470   -0.0433   -0.1913    0.2757    0.5207    0.3134
					-0.0568    0.0198   -0.4483    0.3572    0.2298    0.1598    0.2865   -0.0016    0.2585    0.6637
					
					
					R3alt =
					
					4.2529    0.4174   -1.0723    0.0856    0.2470
					0    2.8470    0.1622   -1.3215    0.4976
					0         0    2.8362    0.5637    1.1207
					0         0         0   -2.4170    0.5911
					0         0         0         0    1.6228
					0         0         0         0         0
					0         0         0         0         0
					0         0         0         0         0
					0         0         0         0         0
					0         0         0         0         0
					
					
					e3alt =
					
					3     5     1     2     4
					
					>> F_rel3alt = norm(Q3alt*R3alt-A3(:,e3alt))/norm(A3)
					
					F_rel3alt =
					
					3.8923e-16
				\end{lstlisting}
			\subsubsection{1000x100 Matrix}
				\begin{lstlisting}
					>> R4 = CompR(B4,D4,p4,k4)
					
					...
					
					>> Q4 = CompQ(B4,p4,k4)
					
					...
					
					
					>> F_rel4 = norm(Q4*R4-A4(:,p4))/norm(A4)
					
					F_rel4 =
					
					2.0532e-15
					
					>> [Q4alt,R4alt,e4alt] = qr(A4,'vector')
					
					...
					
					>> F_rel4alt = norm(Q4alt*R4alt-A4(:,e4alt))/norm(A4)
					
					F_rel4alt =
					
					5.5636e-16
				\end{lstlisting}
			
\end{document}