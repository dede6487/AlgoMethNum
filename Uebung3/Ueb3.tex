\documentclass[11pt,titlepage]{article}

%Laenderspezifische Einstellungen bzgl. Rechtschreibung, Sonderzeichen und Kodierung
\usepackage[utf8]{inputenc}
\usepackage[naustrian]{babel}
\usepackage[T1]{fontenc}
\usepackage{titlesec}
\usepackage{graphicx}
%\usepackage{subcaption}

\usepackage{listings}
\usepackage{color}
\usepackage{courier}
\usepackage{matlab-prettifier}
\definecolor{light-gray}{gray}{0.85}
\lstset{
language=C++,
numbers=left,
style=Matlab-editor,
basicstyle=\mlttfamily,
breaklines=true,
backgroundcolor=\color{light-gray},
tabsize=2,
basicstyle=\footnotesize\ttfamily,
frame=single,
inputencoding=utf8,
extendedchars=true,
showstringspaces=false,
literate =
	{ä}{{\"a}}1
	{ö}{{\"o}}1
	{ü}{{\"u}}1
	{Ä}{{\"A}}1
	{Ö}{{\"O}}1
	{Ü}{{\"U}}1
	{ß}{{\ss}}1
	{ₙ}{{$_n$}}1
}

\def\ContinueLineNumber{\lstset{firstnumber=last}}
\def\StartLineAt#1{\lstset{firstnumber=#1}}

\usepackage[
	a4paper,
	top = 2cm,
	bottom = 2 cm,
	left = 2cm,
	right = 2cm,
	headheight = 15pt,
	includeheadfoot
	]{geometry}
\usepackage{fancyhdr}
\usepackage{amssymb}
\usepackage{amsmath}
\usepackage[english]{varioref}
\usepackage{hyperref}

\fancypagestyle{fancy}{
	\fancyhead[R]{Page \thepage}
	\fancyhead[L]{\leftmark}
	\renewcommand{\headrulewidth}{1.25pt}

	\fancyfoot[L]{\tiny{Alg. Meth. i.d. Num. - Uebung3 , created: \today}}
	\fancyfoot[R]{\tiny{Felix Dreßler (k12105003), Elisabeth Köberle (k12110408)}}
	\cfoot{}
	\renewcommand{\footrulewidth}{1.25pt}
}

\setlength{\headsep}{10mm}
\setlength{\footskip}{10mm}

\setlength{\parindent}{0mm}
\setlength{\parskip}{1.1ex plus0.25ex minus0.25ex}
\setlength{\tabcolsep}{0.2cm} % for the horizontal padding

\pagestyle{fancy}

\title{Algorithmische Methoden in der Numerik - Uebung3}
\author{Felix Dreßler (k12105003) \\ Elisabeth Köberle (k12110408)}
\date{\today} %Erstellungsdatum

\begin{document}
\maketitle

	\section{Aufgabe 4}
		\subsection{Teilaufgabe a}
			\lstinputlisting[]{Aufgabe4/Stiff1.m}
			\lstinputlisting[]{Aufgabe4/Stiff2.m}
			
		\subsection{Teilaufgabe b}
			\lstinputlisting[]{Aufgabe4/RHS.m}
		
		\subsection{Teilaufgabe c}
			Zur Lösung des GLS haben wir zuerst zwei funktionen \emph{f.m} und eine Funktion \emph{ub.m} zur einfacheren Eingabe erstellt.
			
			\lstinputlisting[]{Aufgabe4/f.m}
			\lstinputlisting[]{Aufgabe4/ub.m}
				
			Im Folgenden nun die Lösungen des GLS: (die Matlab Workspace Dateien waren für die Abgabe zu groß, zur Übersichtlichkeit wurden die Ergebnisse der einzelnen Rechnungen großteils unterdrückt)
			Zur besseren Übersicht wurde ein kurzes Skript zur Fehlerberechnung erstellt.
			
			\lstinputlisting[]{Aufgabe4/Test.m}
			
			Die Outputs waren wie folgt:
			
			\begin{lstlisting}
				N
				10
				
				direkte Berechnung :
				Elapsed time is 0.000320 seconds.
				Fehler direkte Lsg ( Maximumsnorm ):
				
				ans =
				
				9.6323
				
				iterative Berechnung 
				pcg converged at iteration 12 to a solution with relative residual 9.4e-07.
				Elapsed time is 0.022338 seconds.
				Fehler iterative Lsg mit ( Maximumsnorm ):
				
				ans =
				
				9.6323
				
				N
				100
				
				direkte Berechnung :
				Elapsed time is 0.019078 seconds.
				Fehler direkte Lsg ( Maximumsnorm ):
				
				ans =
				
				0.0186
				
				iterative Berechnung 
				pcg stopped at iteration 20 without converging to the desired tolerance 1e-06
				because the maximum number of iterations was reached.
				The iterate returned (number 1) has relative residual 0.00052.
				Elapsed time is 0.013484 seconds.
				Fehler iterative Lsg mit ( Maximumsnorm ):
				
				ans =
				
				0.0535
				
				N
				1000
				
				direkte Berechnung :
				Elapsed time is 2.263114 seconds.
				Fehler direkte Lsg ( Maximumsnorm ):
				
				ans =
				
				1.8426e-04
				
				iterative Berechnung 
				pcg stopped at iteration 20 without converging to the desired tolerance 1e-06
				because the maximum number of iterations was reached.
				The iterate returned (number 10) has relative residual 0.0016.
				Elapsed time is 0.375518 seconds.
				Fehler iterative Lsg mit ( Maximumsnorm ):
				
				ans =
				
				0.0368
			\end{lstlisting}
			
			
\newpage			
	\section{Aufgabe 5}
	
		\lstinputlisting[]{Aufgabe5/Jacobi.m}
		
		\subsection{Testberechnung}
			Zur Durchführung dieser Tests wurde wieder ein kurzes Skript erstellt.:
			
			\lstinputlisting[]{Aufgabe5/Test5.m}

			das die folgenden Outputs lieferte:
			
			\begin{lstlisting}
				N:
				10
				
				Elapsed time is 0.161501 seconds.
				pcg converged at iteration 12 to a solution with relative residual 9.4e-07.
				
				ans =
				
				9.6323
				
				
				ans =
				
				1.3104e-05
				
				
			\end{lstlisting}
			
			Ab N = 100 dauerte die Berechnung länger als 2 Minuten, deshalb wurde an diesem Punkt abgebrochen.
			
\end{document}