\documentclass[11pt,titlepage]{article}

%Laenderspezifische Einstellungen bzgl. Rechtschreibung, Sonderzeichen und Kodierung
\usepackage[utf8]{inputenc}
\usepackage[naustrian]{babel}
\usepackage[T1]{fontenc}
\usepackage{titlesec}
\usepackage{graphicx}
%\usepackage{subcaption}

\usepackage{listings}
\usepackage{color}
\usepackage{courier}
\usepackage{matlab-prettifier}
\definecolor{light-gray}{gray}{0.85}
\lstset{
language=C++,
numbers=left,
style=Matlab-editor,
basicstyle=\mlttfamily,
breaklines=true,
backgroundcolor=\color{light-gray},
tabsize=2,
basicstyle=\footnotesize\ttfamily,
frame=single,
inputencoding=utf8,
extendedchars=true,
showstringspaces=false,
literate =
	{ä}{{\"a}}1
	{ö}{{\"o}}1
	{ü}{{\"u}}1
	{Ä}{{\"A}}1
	{Ö}{{\"O}}1
	{Ü}{{\"U}}1
	{ß}{{\ss}}1
	{ₙ}{{$_n$}}1
}

\def\ContinueLineNumber{\lstset{firstnumber=last}}
\def\StartLineAt#1{\lstset{firstnumber=#1}}

\usepackage[
	a4paper,
	top = 2cm,
	bottom = 2 cm,
	left = 2cm,
	right = 2cm,
	headheight = 15pt,
	includeheadfoot
	]{geometry}
\usepackage{fancyhdr}
\usepackage{amssymb}
\usepackage{amsmath}
\usepackage[english]{varioref}
\usepackage{hyperref}

\fancypagestyle{fancy}{
	\fancyhead[R]{Page \thepage}
	\fancyhead[L]{\leftmark}
	\renewcommand{\headrulewidth}{1.25pt}

	\fancyfoot[L]{\tiny{Alg. Meth. i.d. Num. - Uebung1 , created: \today}}
	\fancyfoot[R]{\tiny{Felix Dreßler (k12105003), Elisabeth Köberle (k12110408), Eva Zahn (k12110309)}}
	\cfoot{}
	\renewcommand{\footrulewidth}{1.25pt}
}

\setlength{\headsep}{10mm}
\setlength{\footskip}{10mm}

\setlength{\parindent}{0mm}
\setlength{\parskip}{1.1ex plus0.25ex minus0.25ex}
\setlength{\tabcolsep}{0.2cm} % for the horizontal padding

\pagestyle{fancy}

\title{Algorithmische Methoden in der Numerik - Uebung1}
\author{Felix Dreßler (k12105003) \\ Elisabeth Köberle (k12110408) \\ Eva Zahn (k12110309)}
\date{\today} %Erstellungsdatum

\begin{document}
\maketitle

\section{Funktion - estimateAbsError}

Im folgenden steht die MatLab Funktion zur experimentellen Fehlerabschätzung einer beliebigen Funktion f und einem gegebenen Punkt x0 mit Schrittweite h.

Die Funktion erstellt Matrizen mit den jeweiligen Größen \emph{pxp}, eine mögliche Optimierung dieser Vorgehensweise wäre die Speicherung der Werte in nur einen Vektor der Größe \emph{p}.   

	\lstinputlisting[]{estimateAbsError.m}
		
\section{Tests}
	\subsection{Testfunktionen}
		Obiges Programm wurde mit folgenden Funktionen getestet:
		\lstinputlisting[]{f1.m}
		\lstinputlisting[]{f2.m}
	\subsection{Ergebnisse}
		Untenstehend ist der output des Programs mit den angegebenen Funktionen an den Stellen \\
		\emph{x0} = 0,1 \\
		\emph{x0} = 10\textsuperscript{-3} \\
		\emph{x0} = -1,0000001 \\
		\emph{x0} = 1,0000001
		
		\begin{lstlisting}
			>> x0s = [0.1, 1e-3, -1.0000001, 1.0000001]
			
			x0s =
			
			1.000000000000000e-01     1.000000000000000e-03    
			-1.000000100000000e+00     1.000000100000000e+00
			
			>> arrayfun(@(x0) estimateAbsError(@f1,x0),x0s)
			
			ans =
			
			1.326970321773750e-16     9.930136612989092e-17     
			5.307881287095000e-17     4.784171007815724e-11
			
			>> arrayfun(@(x0) estimateAbsError(@f2,x0),x0s)
			
			ans =
			
			1.592364386128500e-16     9.930136612989092e-17     
			7.829083830833991e-18     1.061576257419000e-16
		\end{lstlisting}


\end{document}